%%%%%%%%%%%%%%%%%%%%%%%%%%%%%%%%%%%%%%%%%
% Medium Length Professional CV
% LaTeX Template
% Version 2.0 (8/5/13)
%
% This template has been downloaded from:
% http://www.LaTeXTemplates.com
%
% Original author:
% Trey Hunner (http://www.treyhunner.com/)
%
% Important note:
% This template requires the resume.cls file to be in the same directory as the
% .tex file. The resume.cls file provides the resume style used for structuring the
% document.
%
%%%%%%%%%%%%%%%%%%%%%%%%%%%%%%%%%%%%%%%%%

%----------------------------------------------------------------------------------------
%	PACKAGES AND OTHER DOCUMENT CONFIGURATIONS
%----------------------------------------------------------------------------------------

\documentclass{resume} % Use the custom resume.cls style
\usepackage{enumitem}
\pagenumbering{gobble}
\usepackage{color}
\usepackage{soul}
\usepackage{multicol}
\newcommand*{\MAGMA}{\textsc{magma}\xspace}
\setlength\parindent{0pt}
\usepackage[left=0.75in,top=0.6in,right=0.75in,bottom=0.6in]{geometry} % Document margins

\name{Veronica Kelsey} % Your name
\address{ veronicakelsey@live.com} % Your phone number and email

\begin{document}

%----------------------------------------------------------------------------------------
%	EDUCATION SECTION
%----------------------------------------------------------------------------------------

\begin{rSection}{Education}

 \textbf{PhD Mathematics - University of St Andrews -} \textbf{2018 - Present} \\
Supervised by Prof Colva M. Roney-Dougal and Dr Martyn Quick.\\
My research is on characterising when a maximal subgroup of a finite simple group is a maximal coclique in the group's generating graph. Currently focusing on the case of alternating groups.\\
 \\
 \textbf{MMath Mathematics - The University of Manchester -} \textbf{2014 - 2018} \\
Average -  98$\%$\\
Dissertation - Mathieu groups and Chamber Graphs - Grade $98 \%$ - Supervised by Prof Peter Rowley\\
Third Year Project - Representations of Quivers - Grace $89 \%$ - Supervised by Prof Mike Prest
\end{rSection}
\vspace{0.3cm}
 \begin{rSection}{Publications}
$\mathbf{M_{24}}$\textbf{ Orbits of Octad Triples (with P.J. Rowley)}\\ \textit{Graphs and Combinatorics}, Volume 34, (2018) Issue 6, pp 1429 - 1443 \\
\textbf{Chamber Graphs of some Geometries that are Almost Buildings (with P.J. Rowley)} \\\textit{
Innovations in Incidence Geometry}, to appear\\
\textbf{A Note on Involution Centralizers in Black Box Groups (with P.J. Rowley)}\\ Submitted. MIMS EPrint: 2018.21 \\
\textbf{Chamber Graphs of Minimal Parabolic Sporadic Geometries (with P.J. Rowley)}\\ In preparation
\end{rSection}

\vspace{0.3cm}
\begin{rSection}{Research Experience}
\textbf{London Mathematical Society Undergraduate Researcher 2017} \\
An LMS funded summer research project with Prof Peter J Rowley. We looked at certain finite simple groups and the combinatorial objects they act upon. This work resulted in two published papers.
\end{rSection}

\vspace{0.3cm}

\begin{rSection}{Awards and Scholarships}
\textbf{Outstanding Academic Achievement Award 2018} - Presented to $0.5 \%$ of the Manchester student population \\
\textbf{Institute of Mathematics and its Applications (IMA) Prize 2018} - Final year outstanding project\\
\textbf{President's Doctoral Scholar Award (Manchester) 2018/19} - As part of a PhD offer (declined)\\
\textbf{Nominated for The Distinguished Achievement Award 2018} - Undergraduate of the Year\\
\textbf{Fourth Year Scholarship 2017} - Tuition fee waiver \\
\textbf{LMS Undergraduate Research Bursary 2017} - Funding for a summer research project\\
\textbf{BP achievement award 2016} - For an essay on Big Data\\
\textbf{Dalton scholarship 2016} - Academic achievement in 2\textsuperscript{nd} year\\
\textbf{Dalton scholarship 2015} -  Academic achievement in 1\textsuperscript{st} year\\
\textbf{Manchester Bursary 2014} - Academic achievement at A-level
\end{rSection}
\vspace{0.3cm}
%----------------------------------------------------------------------------------------
%	WORK EXPERIENCE SECTION
%----------------------------------------------------------------------------------------
\newpage
\begin{rSection}{Talks}
\textbf{A Brief Introduction to $\mathbf{M_{12}}$}\\ Pure Postgraduate Seminar, St Andrews, 2018\\
\textbf{The Kitten, The MINIMOG and $\mathbf{M_{12}}$}\\ Pure Postgraduate Seminar, Manchester, 2018
\end{rSection}

\vspace{0.3cm}
\begin{rSection}{Teaching }
\textbf{Linear Maths} tutor, St Andrews - Present\\
\textbf{Abstract Algebra} tutor, St Andrews - Present\\
\textbf{Linear Algebra} tutor, Manchester - 2018\\
\textbf{Foundations of pure mathematics} tutor, Manchester -2017\\
\textbf{Linear Algebra} PASS leader, Manchester, 2016\\
\textbf{Sequences and Series} PASS leader, Manchester, 2016\\ 
\textbf{Introduction to Statistics, PASS leader} PASS leader, Manchester, 2016\\
\textbf{Probability 1} PASS leader, Manchester, 2015\\
\textbf{Sets, Numbers and Functions} PASS leader, Manchester, 2015\\
\\
PASS is a University of Manchester scheme where students organise and lead twice weekly tutorials for first years.
\end{rSection}
\vspace{0.3cm}
\begin{rSection}{Conferences Attended}
\textbf{Workshop on Groups, Generalisations and Applications} Aberdeen, January 2019\\
\textbf{The Functor Categories for Groups} on Graphs and groups Lancaster, December 2018\\
\textbf{The Manchester-Birmingham-London-Bristol ``Triangle"} on Finite Groups Manchester, May 2018\\
\textbf{ LMS Prospects in Mathematics Workshop} - Reading, September 2017
\end{rSection}
\vspace{0.3cm}
\begin{rSection}{Other Activities}
\textbf{Topical Secretary for MathSoc - 2017/18}\\
I organised and ran talks on interesting topics in maths for undergraduates and staff. This encouraged students' engagement with maths outside their studies. Talks were well attended with approximately 50 people at each talk and there was positive feedback from all participants. 
\end{rSection}
\vspace{0.3cm}
\begin{rSection}{References}
\begin{multicols}{2}
Prof Mike Prest - Project Supervisor\\ mike.prest@manchester.ac.uk\\
\\
Prof Alexandre Borovik - Dissertation Marker \\ 
alexandre.borovik@manchester.ac.uk
\end{multicols}
\end{rSection}
%----------------------------------------------------------------------------------------
%	EXAMPLE SECTION
%----------------------------------------------------------------------------------------

%\begin{rSection}{Section Name}

%Section content\ldots

%\end{rSection}

%----------------------------------------------------------------------------------------

\end{document}
